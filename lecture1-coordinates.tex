\subsection{Coordinates in Projective Spaces}
In any given vector space $V$ with $\text{dim $V$}=n+1$ we can define a basis of $n+1$-many linearly independent vectors $v_0,v_1,...,v_n$ that span the space. We can define something similar in the projective space of the associated vector space. This allows us to uniquely identify any arbitrary vector $v\in V$ as $v \coloneq \sum_{i=0}^n {\lambda_iv_i} \text{ with } \lambda_i \in \F$.

%coordinate definition
\begin{definitionBox}{Homogenous Coordinates}
    \begin{itemize}
        \item Let $\lambda_0, ..., \lambda_n$ be not all zero.
        \item Let $p \in \PSV$ be a point (we call the 1-dimensional set of vector $[v]$ for which we can pick a representive $v \in [v]$ from $V$ a point because vectors in $[v]$ point towards this "point in infinity").
    \end{itemize}
    We can now write $[\lambda_0,\lambda_1,...,\lambda_n]$ for the given point $p = \lambda_0v_0+\lambda_1v_1+...+\lambda_nv_n$. \\
    Any arbitrary point $p\in PSV$ can be expressed using these \textbf{homogenous coordinates}.
\end{definitionBox}