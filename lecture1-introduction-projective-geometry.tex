\subsection{Introduction to Projective Geometry}

Projective spaces are a generalisation of spaces such as $\mathbb{R}^n$ using linear algebra. \\




\begin{definitionBox}{Projective Space}    
    %requirements
    \begin{itemize}
        \item Let $\F$ be a field (usually, these fields are $\R$ or $\C$ but they can by also be any arbitrary field $\F_p$).
        \item Let $V$ be a finite-dimensional vector space over the given field $\F$. We usually assume that $V$ has $\mathrm{dim}(V) = n +1$ with $n \in \N$.
    \end{itemize}  

    %definition
    We define the projective space $\PS(V)$ associated to $V$ to be the \textbf{set of 1-dimensional vector subspaces} $U_i$, $i \in \N$ with $U \leq V$.
\end{definitionBox}
\vspace{0.5cm}



%further notation for vector scaling
We now introduce further notation for simplified notation of scaling vectors.

\begin{definitionBox}{Set of Scaled Vectors}
    For each $0 \neq v \in V$ we write $[v] = \{\lambda v: \lambda \in \F\}$.
    $[v]$ thus becomes a direction and a base vector in this 1-dimensional vector subspace.
\end{definitionBox}
\vspace{0.5cm}



%defining P(V) with regards to [v]
Now, using this new found notation we can actually mathematically describe our set of 1-dimensional vector subspaces which in turn all consist of $[v]$ over our given vector space $V$.

\begin{definitionBox}{$\PS(V)$ with regards to $[v]$}
    We can define $\PS(V)$ with regards to every non-zero $v \in V$ over a given field $\F$
    \begin{gather*}
        \PSV = \{[v]: 0 \neq v \in V\}
    \end{gather*}
    We can now further deduce from $[v]$ that it cannot be unique so that it holds $[v] = [w]$ iff $v = \lambda w, \: \lambda \in \F \backslash \{0\}$
\end{definitionBox}

%understanding the dimension and creating intuition
This set of vectors $[v]$ now allows to understand the projective space of given vector space $V$ as a set of 1-dimensional directions which are unique except for their scaling. \\
The introduction of a projection space $\PS(V)$ based on a given vector space $V$ leads to the reduction of a dimension because vectors of the given vector space $V$ that go into the same direction as are grouped together. The are considered to be identical except for the scaling factor. This scaling factor is thrown out. This leads to the reduction in dimension. \\
Thus, we can express via group division which leads to a reduction in its dimension. Here, we turn our field $\F$ into a 1 dimensional group $\F^*$ via the $*$ operator.

\begin{definitionBox}{$\PSV$ via group division}
    %requirements
    \begin{itemize}
        \item Let $F^*$ the 1-dimensional group which means $\PSV$ is both compact and a Hausdorff space (e.g. two distinct points can be "housed off" via disjointed open sets).
    \end{itemize}
    \begin{gather*}
        \PSV = (V \backslash \{ 0 \}) /\F^*
    \end{gather*}
\end{definitionBox}

%on mapping it to other topological spaces
This means $\PSV$ is a topological manifold, i.e. it behaves locally like a Euclidean space. Thus, for every point $v \in \PSV$ a small neighbourhood of points can be mapped to the the corresponding vector space $V$ over $\F$. \\

%dimensions of P(V) and V
We can thus observe the correspondence of dimensions between $V$ and $\PSV$
\begin{definitionBox}{Dimension of $\PSV$}
    \begin{itemize}
        \item Let $\F^*$ be the group over the field $\F$ with $\mathrm{dim}\;\F = 1$
    \end{itemize}
    \begin{gather*}
        \mathrm{dim}\; \PSV = \mathrm{dim}\; V -1
    \end{gather*}
\end{definitionBox}




This is particularly illustrative if you consider an $n$ dimensional vector space $V_p$ over an arbitrary field $\F$ with vectors $v\in V$ with $v_i \in \F$ expressed using generalised, $n$-dimensional polar coordinates. Thus, $v$ is expressed via $v_1$ giving the length of the vector and $v_i, i \in [2,n]$ specifying the angle. If you then consider vectors of the same angle following the same direction towards a point in infinity and you are only concerned with retaining the direction and do not need the scaling factor you can omit $v_1$, thus reducing the dimension and gaining $\mathrm{dim}(\PS(V)) = \mathrm{dim}(V) - 1$. 


%example in in 3 dimensional polar coordinates
\begin{exampleBox}{Projective Space in $\R^3$}
    This example illustrates how the projective space of $V = \R^3$ $\PS(V) = \PS(\R^3)$ reduces its dimension due to the omission of scaling.
    %giving the requirements
    \begin{itemize}
        \item Let $V = \R^3$ (thus, $\F = \R$) and assume each $v \in V$ is represented using a transformation from Cartesian coordinates to angles and a radius.
        \item Every $v \in V$ can be expressed as and then transformed to a vector $v' \in V$ with a spherical basis: 
            \begin{gather*}
                v = (x,y,z) \Rightarrow v' =  (r, \vartheta,\varphi)  \\
                r = \sqrt{x^2+y^2+z^2} \\
                \vartheta = arctan(y/x) \\
                \varphi = arccos(z/r) \\   
            \end{gather*}
    \end{itemize}  
    
    % conclusions drawn based on examplary vectors
    We can now observe multiple things using the exemplary vectors $v_1 = (1, \pi/2,\pi/2)$ and $v_2 = (2, \pi,\pi)$
    \begin{enumerate}
        \item We can find $\lambda = 2 \in \R$ to express $\lambda v_1 = v_2$. Thus, these vectors are linearly dependent.
        \item We realise that if we omit the length of each vector expressed in $v_i^1, i \in [1,2]$, both point into the same direction towards a point in infinity.
        \item If we know want to express location of the point (the direction) we only require the two angles. We have thus reduced the dimension by one.
    \end{enumerate}
    If we now want to generalise these findings we can observe that both $v_1 \in [v]$ and $v_2 \in [v]$ because
    \begin{gather*}
        \PS(V) = \{ [v] :  0 \neq v\in V\} = \{ \{v_1, v_2, ...\}\} 
    \end{gather*}
    
    %examplary vectors set into the 1 dimensional set
\end{exampleBox}